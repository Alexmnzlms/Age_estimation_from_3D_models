\chapter{Conclusiones}
\label{chap:conclusiones}

A lo largo de este TFG, establecemos el problema de la estimación del PB dentro del marco de la AF como uno de los principales retos dentro de este campo y constatamos la importancia de dotar a los antropólogos forenses con sistemas de apoyo que ayuden a reducir los errores cometidos por el experto humano en la toma de decisiones en el proceso de estimación de PB, en concreto de la estimación de la edad a partir de la sínfisis púbica.\\

En base a los resultados obtenidos, el problema de \textbf{estimar la edad mediante la aplicación de técnicas de DL a modelos 3D de la sínfisis púbica} ha sido abordado y resuelto con éxito. Nos encontramos ante un problema complejo, que a priori presentaba grandes retos, dado el limitado número de modelos 3D disponibles para el entrenamiento y el desbalanceo del conjunto de datos. Ha sido necesario encontrar e implementar una representación de datos 3D para uso en modelos de DL y además ha sido necesario diseñar e implementar modelos capaces de hacer un buen uso de esta representación.

Todos los objetivos definidos en este TFG se han cumplido en su totalidad. Se ha realizado un estudio pormenorizado y sistematizado del estado del arte de la representación de modelos 3D y se ha elegido una representación adecuada para los modelos 3D a nuestra disposición. Se han desarrollado las herramientas necesarias para implementar esta representación y, además, estas están disponibles de forma libre a disposición de cualquiera en Github (\url{https://github.com/Alexmnzlms/Age_estimation_from_3D_models}). Se han diseñado modelos profundos convolucionales que han probado ser eficaces a la hora de resolver la tarea de estimar la edad. Se han aplicado estrategias para paliar la falta de datos y el desbalanceo de clases, y estas han probado ser estrategias eficaces. Y, para terminar, se ha comprobado que los modelos desarrollados, en combinación con la representación escogida, han sido capaces de aprender de forma automática las características representativas de los restos óseos.\\

Finalmente se han obtenido resultados que igualan o incluso mejoran a los presentes en el estado del arte, por lo que este trabajo se presenta como una opción muy competitiva dentro del campo de la estimación de la edad a partir de las sínfisis púbicas. Igualamos al mejor resultado del estado del arte, aportado por \cite{villegas_TFG}, con un MAE de 8,55. Respecto al RMSE, quedamos por encima de \cite{villegas_TFG} por una diferencia menor a 0,5 años. La principal diferencia entre ambos métodos, es que mientras que \cite{villegas_TFG} realiza una estimación basada en componentes e involucra la participación de un experto forense en la extracción de características, nuestra propuesta extrae las características morfológicas de las sínfisis púbicas de forma automática a partir del modelo 3D en su totalidad.
%¿cuánto mejor somos que el estado del arte? 
Constatamos que las características extraídas por nuestro modelo a partir de los modelos 3D, a priori, parecen ser de la misma calidad que las características morfológicas de las sínfisis púbicas utilizadas en los métodos propuestos en el estado del arte. La mejor estimación obtenida por nuestro modelo, se enmarca dentro del rango de individuos adultos (entre 19 y 64 años) con una media de un MAE de 7,7 y un RMSE de 9,96 en el conjunto de test aumentado (ver Tabla \ref{tab:r3_c_total}) y un MAE de 7,66 y un RMSE de 9,74 en el conjunto de test no aumentado (ver Tabla \ref{tab:r3_s_total}). Dentro de este rango, vemos como las mejores estimaciones obtenidas para cada una de las fases propuestas por Todd (fases 4 a 9) se encuentran por debajo de 10 años de error tanto en MAE como en RMSE. Seguido de esta, la segunda mejor estimación se da en el grupo de los niños. Nuestro modelo obtiene un MAE de 9,6 y un RMSE de 12,98 en el conjunto de test aumentado y un MAE de 9,26 y un RMSE de 11,36 en el conjunto de test no aumentado. Destaca la calidad del ajuste dado la escasez de datos pertenecientes a este grupo. Para terminar, la peor estimación se da en el rango de los ancianos, dado que este el grupo más reducido de todos. Nuestro modelo no obtiene estimaciones con un error inferior a 25 años. Esto puede estar motivado por la falta de datos y por la dificultad de estimar y extraer características morfológicas discriminatorias a partir de muestras de individuos de avanzada edad.
%¿qué rangos de edad estimamos mejor? 
\\

Nuestro modelo presenta la principal ventaja de ser automático, es decir, no requiere la presencia de un experto antropólogo para realizar la estimación. Una vez obtenida la representación 3D de las sínfisis púbicas, los métodos propuestos pueden ser aplicados de forma directa y efectiva. A su vez, dado que aplica técnicas y algoritmos de DL clásicos, es perfectamente escalable y aplicable a nuevos conjuntos de datos. Sin embargo, esta propuesta no esta exenta de desventajas, ya que no se extrae ninguna información de utilidad para el antropólogo forense más allá de la estimación de la edad de la muestra. Aunque este no era el objetivo del enfoque propuesto, varios de los métodos presenten en el estado del arte pueden servir para validar la utilidad de utilizar ciertas regiones de la sínfisis para la estimación u otras en base a los resultados.
%¿qué pros y contras tiene nuestro modelo en relación con el estado del arte? 

\section{Trabajos futuros}

Aunque este proyecto haya concluido, existen varias líneas que seguir para mejorar el desempeño del sistema en el futuro y comprender de mejor manera el proceso de aprendizaje que ha realizado.\\

En primer lugar, una expansión de este proyecto podría ser estudiar nuevas formas de combinar los outputs de los modelos más allá de realizar una media aritmética (combinación de las salidas de los clasificadores, weak learners, en un ensemble de clasificadores \cite{cao2020ensemble, ganaie2021ensemble}). El problema podría enfocarse incluso como un reto que el modelo debe resolver, en el sentido de que el propio modelo podría aprender de forma empírica cuál de las imágenes empleadas como input aporta más información y beneficia más al aprendizaje.\\

Otra línea de trabajo podría ser dotar a las redes convolucionales consideradas de una componente explicable \cite{arrieta2020explainable,angelov2020towards,choo2018visual}, por ejemplo, estudiando las regiones de activación de los modelos. Para una red CNN cualquiera, es posible visualizar sus mapas de activación para identificar aquellas regiones de la imagen que han tenido un mayor peso a la hora de realizar una predicción determinada \cite{selvaraju2017grad}. Además como el input de las redes profundas son proyecciones, estos mapas de activación podrían proyectarse sobre el modelo 3D de forma inversa a la proyección actual, de forma que se obtuviera un mapa 3D de las zonas del hueso que están siendo utilizadas para realizar la discriminación. Esto permitiría a los antropólogos forenses y expertos comparar estas regiones de activación con otras regiones características propuestas en la literatura (por ejemplo las regiones características propuestas por Todd) de forma que se pudiera corroborar la eficacia de estas regiones o incluso descubrir nuevas regiones características.
