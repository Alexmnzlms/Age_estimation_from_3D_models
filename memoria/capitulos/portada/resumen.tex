


La estimación de la edad es una de las tareas más relevantes dentro del campo de la antropología forense y resulta de utilidad tanto para la identificación de personas vivas como muertas en problemáticas tan distintas como desapariciones de personas, crisis migratorias, catástrofes naturales, crímenes sin resolver, etc. 
Entre las diversas técnicas existentes para estimar la edad de la muerte en individuos fallecidos, destacan el estudio de la características morfológicas presentes en la sínfisis del pubis, método propuesto por T. W. Todd en 1921 y utilizado aún a día de hoy. Este método presenta un gran nivel de subjetividad, ya que el estudio de estas características depende casi en su totalidad de la experiencia del antropólogo forense. Destaca por tanto, el bajo nivel de aplicación de técnicas automáticas existentes para el apoyo de esta tarea.


Este TFG aborda el desarrollo de un sistema capaz de estimar la edad de un individuo a partir de un modelo 3D de su sínfisis púbica. Nuestra propuesta presenta un enfoque totalmente nuevo y transgresor respecto a las técnicas actuales. A diferencia de estas, donde es necesaria la participación de un antropólogo forense experto para extraer los rasgos morfológicos de las muestras, el método que proponemos será capaz de extraer estos rasgos de forma automática. Conseguimos este objetivo enfocando este problema dentro del paradigma del aprendizaje profundo. Para poder aplicar técnicas relativas a este campo utilizamos un método de proyección panorámica que permite transformar los modelos 3D en imágenes 2D manteniendo la información morfológica de los mismos. Para entrenar los modelos profundos convolucionales propuestos utilizamos el conjunto de 1104 modelos 3D de sínfisis púbicas propiedad de la Universidad de Granada. Este conjunto presenta las limitaciones de ser demasiado limitado en número y poseer clases altamente desbalanceadas, por lo que hemos aplicado estrategias de aumento y balanceo de datos.

Los resultados obtenidos por el método propuesto son comparables a los del estado del arte actual, por lo que el modelo obtenido es altamente competitivo y presenta una alternativa viable para el apoyo en la estimación de la edad de un individuo.